%%%%%%%%%%%%%%%%%%%%%%%%%%
%% global configuration %%
%%%%%%%%%%%%%%%%%%%%%%%%%%

%the overline configuration on the pages-----------
%mainly refer to "packages.tex" in https://www.overleaf.com/latex/templates/tesi-unitn/jjrxgvzgxpsm
\usepackage{fancyhdr}
\pagestyle{fancy}
\fancyhf{}
\fancyhead[RO,LE]{Road Network Similarity Metrics}
\fancyfoot[LE,RO]{\thepage} %the position of page number
%the overline configuration on the pages-----------
%--------------------------------------------------

%clear the empty double page-----------------------
%refer to https://texfaq.org/FAQ-reallyblank
\let\origdoublepage\cleardoublepage
\newcommand{\clearemptydoublepage}{%
    \clearpage{\pagestyle{empty}\origdoublepage}%
    }
\let\cleardoublepage\clearemptydoublepage
%clear the empty double page-----------------------
%--------------------------------------------------


%make the reference clickable----------------------
\usepackage{hyperref}
\hypersetup{  
    colorlinks=true, 
    linkcolor=black, 
    citecolor=red, 
    urlcolor=blue  
    }
%make the reference clickable----------------------
%--------------------------------------------------


%bibliography--------------------------------------
%refer to https://www.overleaf.com/learn/latex/Bibliography_management_in_LaTeX
\usepackage[
    backend=biber,
    style=alphabetic,
    sorting=ynt
    ]{biblatex}
\addbibresource{biblitex.bib} %Import the bibliography file
\usepackage{url} %make the online sites clickable
%bibliography--------------------------------------
%--------------------------------------------------


%colors&figures------------------------------------
\usepackage{geometry} % adjust the pages configuration
\usepackage{graphicx} % insert figures
\usepackage[export]{adjustbox} % for title page
\usepackage{xcolor}
\usepackage{csvsimple}
\usepackage{float} %used to place figure with specifier [H]
%colors&figures------------------------------------
%--------------------------------------------------

\usepackage{lscape}
\usepackage[table,xcdraw]{xcolor}
%others--------------------------------------------
\usepackage{enumitem} % to use "description" environment
%others--------------------------------------------
%--------------------------------------------------

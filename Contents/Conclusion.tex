%%%%%%%%%%%%%%%
% Conclusion %
%%%%%%%%%%%%%%%

In summary, this was the first study to compare the performance of various graph-based similarity algorithms on two different road network datasets. Furthermore, each of the methods used was compared to one another in order to identify similarities.

The results of the similarity measure between the different road networks, as well as the correlations between each method used with the Kendall tau distance, varied depending on the type of road network used for the comparison. An intriguing discovery was that the clustering algorithm was able to identify clusters of road networks that have similar structures of the road networks as graphs, which goes slightly against the initial approach of identifying road networks with similar patterns such as Grid, Radial, Tree, Linear, or Cul De Sac as the algorithms themselves do not take the patterns of the road networks into consideration when comparing them but the topological characteristics which are stored in the graph structure.

However, a careful examination of the dendrograms that identified clusters among the road networks reveals that specific road networks with similar patterns belonged to the same cluster. For example, the road networks, Champs-Elysees, Paris, Amsterdam, and City of Westminster, London all belonged to the same cluster with shorter distances, as did Toronto-Canada, Ejecutivo-Mexico City, and the District of Columbia-USA. Each similarity measure behaved differently as expected from the correlation to identify similarities in the similarity methods because they operated at different levels and used different comparisons. Despite this, there are strong correlations between cosine similarity, Euclidean distance, Jaccard distance, and Jaccard similarity.

Since this study is only the beginning of the research into identifying similarities on road networks as graphs. First, studies comparing alternative methods to those presented in section 3 should be conducted. In addition, the method for selecting road networks should be investigated. Following that, a more in-depth analysis of the results of this study is required to provide precise conclusions on the usefulness of the similarity measures in this context, followed by more appropriate validation of the clustering results.

( \cite{Agryzkov:2012}, \cite{Barthelemy:2004}, \cite{Mheich:2018})

\section{Future Directions}
Since this study is only the beginning of the research into identifying similarities on road networks as graphs. From a methodological viewpoint, first, more efforts into studies that compare alternative methods to those presented in section 3 should be conducted. In addition, the method for selecting road networks should be investigated. Following that, a more in-depth analysis of the results of this study is required to provide precise conclusions on the usefulness of the similarity measures in this context, followed by more appropriate validation of the clustering results. Finally, more work is needed to develop and optimize similarity algorithms that combine multiple graph characteristics into a single similarity score. Before applying these algorithms to real-world road networks, they should be analyzed/validated using simulated data (where ground truth can be obtained to some extent). Another methodological approach is to decipher how road networks change over time by using similarity methods in dynamic road algorithms. Most dynamic analysis algorithms require a similarity/correlation step to compare adjacent networks. Typically, the classical correlation coefficient is used (O'Neill et al., 2018). The incorporation of a network-based similarity index into dynamic road algorithms can significantly improve their performance.
The expectation is that this study will inspire more researchers to contribute ideas other than those described above in the field of road network similarity, either methodologically or practically.
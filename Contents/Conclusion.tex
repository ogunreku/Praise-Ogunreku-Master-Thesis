%%%%%%%%%%%%%%%
% Conclusion %
%%%%%%%%%%%%%%%

As long as there are functional/structural road networks, there will be people looking for comparisons between them. Here, road networks have been compared so also the main methods and algorithms that can be used to compare them.

Which method then? The answer depends on the application itself. If the objective is to reveal statistical difference between two groups then the methods based on the graph theoretical approach (node-wide or edge-wise) can be good candidates provided that the hypothesis are well set and statistical parameters are carefully chosen (correction for multiple comparisons, for instance).

However, if the objective is to produce a similarity score (usually normalized between 0 and 1), then the distance-based graph comparison methods are more appropriate. Validation of algorithms, like the comparative analysis of Mheich et al. (2018), can allow us to identify a set of methods that perform well on such networks.
Therefore, there is no guarantee that methods performing well on benchmarks also give reliable results. \cite{Agryzkov:2012}

\section{Some Challenges in Road Network Similarity}
First, in the statistical comparison (graph theoretical–based approach), the major difficulty arises from the fact that graph measures depend on the number of nodes and edges. To compare two different road networks, choosing equal size and density has become more popular so that differences in graph measures appear solely through structural changes (van Wijk, Stam, & Daffertshofer, 2010). However, this can be only achieved by taking a fixed number of nodes and imposing a desired average degree by adjusting the binary threshold (van Wijk et al., 2010).

Second, knowing the graph metrics that enable one to detect the difference between road networks is not obvious. The choice of this graph metric is often empirical. For a more appropriate approach, these graph metrics should be driven by adopting methods based on Network Representation Learning (NRL). Indeed, these NRL approaches avoid the necessity for thorough feature engineering and have led to very important results in network-based tasks, such as node classification, node clustering, and prediction (D. Zhang, Yin, Zhu, & Zhang, 2018). From the analysis and the results, the believe is that the NRL could be very useful to the network community for the adaption of representation learning techniques to specific applications that are of interest in the field. In addition, a key challenge is to encapsulate several graph metrics in one similarity score that can describe the difference between two graphs without missing any characteristics.
Third, the distance-based graph algorithms developed in different fields are indeed very useful for detecting uncaptured characteristics by graph metrics. However, these algorithms are still limited to undirected graphs. More efforts are needed to expand these algorithms to deal with directed (causal) road networks.


\section{Future Directions}
From a methodological viewpoint, more efforts are needed to develop and optimize similarity algorithms that combine several graph characteristics into one similarity score. These algorithms should be first analyzed/validated using simulated data (where ground truth can be, to some extent, obtained) before applying them to real road networks. Another methodological approach is to use similarity methods in road dynamic algorithms in order to decipher how road networks change over time. In most dynamic analysis algorithms, a similarity/correlation step is always needed to compare adjacent networks. This is usually done by using the classical correlation coefficient (O’Neill et al., 2018). Adding a network-based similarity index into road dynamic algorithms can dramatically improve their performance.
The hope is that this study will motivate more researchers to contribute with other ideas than those described above in the road network similarity field, from a methodological and/or an applicative perspective.
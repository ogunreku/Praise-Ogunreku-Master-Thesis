%%%%%%%%%%%%%%%
% Conclusion %
%%%%%%%%%%%%%%%

In summary, this was the first study to compare the performance of various graph-based similarity algorithms on two different road network data sets. Furthermore, each of the methods used was compared to one another in order to identify similarities.

The results of the similarity measure between the different road networks, as well as the correlations between each method used with the Kendall tau distance, varied depending on the type of road network used for the comparison. An intriguing discovery was that the clustering algorithm was able to identify clusters of road networks that have similar structures of the road networks as graphs, which goes slightly against the initial approach of identifying road networks with similar patterns such as Grid, Radial, Tree, Linear, or Cul De Sac as the algorithms themselves do not take the patterns of the road networks into consideration when comparing them but the topological characteristics which are stored in the graph structure.

A close examination of the dendrograms that identified clusters among the road networks, on the other hand, reveals that specific road networks with similar patterns belonged to the same cluster. The road networks of Champs-Elysees in Paris, Amsterdam, and the City of Westminster in London, for example, all belonged to the same cluster with shorter distances, as did Toronto-Canada, Ejecutivo-Mexico City, and the District of Columbia-USA. Because they operated at different levels and used different comparisons, each similarity measure behaved differently, as expected from the correlation to identify similarities in the similarity methods. Despite this, cosine similarity, Euclidean distance, Jaccard distance, and Jaccard similarity have strong correlations.


\section{Future Directions}
Because this research into identifying similarities on road networks as graphs is still in its early stages. From a methodological standpoint, more efforts should be made to conduct studies that compare alternative methods to those presented in section 3. Furthermore, the method for selecting road networks should be researched. Following that, a more in-depth examination of the study's findings is required to provide precise conclusions on the utility of similarity measures in this context, followed by more appropriate validation of the clustering results. Finally, more work is required to develop and optimize similarity algorithms that combine multiple graph properties into a single similarity score. Before applying these algorithms to real-world road networks, they should be tested and validated on simulated data (where ground truth can be obtained to some extent). Another methodological approach is to use similarity methods in dynamic road algorithms to decipher how road networks change over time. To compare adjacent networks, most dynamic analysis algorithms require a similarity/correlation step. The classical correlation coefficient is commonly used [\cite{ONeill:2018}]. The incorporation of a network-based similarity index into dynamic road algorithms can improve their performance significantly.
It is hoped that this study will inspire other researchers to contribute ideas in the field of road network similarity that are not described above, either methodologically or practically.
%%%%%%%%%%%%
% Zusammenfassung %
%%%%%%%%%%%%

Das Verständnis der Ähnlichkeiten von Straßennetzstrukturen in Städten ist eine schwierige, aber notwendige Aufgabe für die Gestaltung besserer städtischer Umgebungen. Die Entdeckung und der Vergleich von Strukturen wie Graphen, modularen Gemeinschaften, reichen Clubs, Knotenpunkten und Bäumen liefern Informationen über die räumlichen und zeitlichen Eigenschaften von Straßennetzen.

Mit den jüngsten Fortschritten in der GIS- und Graphentheorie, der raschen Zunahme der verfügbaren Daten und der Flexibilität der Netzwerkmodellierung ist das Problem der Entwicklung wirksamer quantitativer und garantierter A-priori-Begriffe von Metriken für den Vergleich von Straßennetzen aufgetaucht. Für diese Aufgabe gibt es zahlreiche Metriken, von denen die meisten in verschiedenen Bereichen wie Bioinformatik, Cybersicherheit und Analyse sozialer Netzwerke angewandt werden. Es wurde jedoch noch keine vergleichende Studie durchgeführt, um die Wirksamkeit dieser Metriken bei der Unterscheidung gemeinsamer Straßennetztopologien auf verschiedenen Strukturskalen zu ermitteln.

Ziel dieser Arbeit ist es, die am häufigsten verwendeten Metriken und Distanzmaße zu untersuchen und zu vergleichen und ihre Fähigkeit zu bewerten, zwischen den in Straßennetzen vorkommenden topologischen Merkmalen zu unterscheiden. Schließlich werden die Metriken auf einen neuen spezialisierten Datensatz angewandt, der aus graphischen Darstellungen von Straßennetzen in 20 Städten besteht. Um die Ergebnisse dieser Art von Analyse zu erörtern, enthalten die Datensätze eine Sammlung von leicht identifizierbaren Straßennetzmustern.
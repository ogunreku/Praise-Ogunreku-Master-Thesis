%%%%%%%%%%%%
% Zusammenfassung %
%%%%%%%%%%%%

Das Verständnis der Ähnlichkeiten von Straßennetzstrukturen in Städten ist eine schwierige, aber notwendige Aufgabe für die Gestaltung besserer städtischer Umgebungen. Die Entdeckung und der Vergleich von Strukturen wie Graphen, modularen Gemeinschaften, reichen Clubs, Knotenpunkten und Bäumen liefern Informationen über die räumlichen und zeitlichen Eigenschaften von Straßennetzen.

Mit den jüngsten Fortschritten in der GIS- und Graphentheorie, der schnellen Zunahme verfügbarer Daten und der Flexibilität der Netzwerkmodellierung ist die Herausforderung entstanden, effiziente quantitative und garantierte Leistungsprioritäten zum Vergleichen von Straßennetzwerken zu entwickeln. Für diese Aufgabe gibt es viele Metriken, von denen die meisten in verschiedenen Bereichen wie Bioinformatik, Cybersicherheit und Analyse sozialer Netzwerke angewendet werden. Es wurde jedoch noch keine vergleichende Studie durchgeführt, um die Wirksamkeit dieser Metriken bei der Unterscheidung gängiger Straßennetztopologien auf verschiedenen strukturellen Maßstäben zu bestimmen.

Das Ziel dieses Papiers ist es, die am häufigsten verwendeten Entfernungsmetriken und -maße zu untersuchen und zu vergleichen und ihre Fähigkeit zu bewerten, zwischen topologischen Merkmalen in Straßennetzen zu unterscheiden. Schließlich werden die Metriken auf einen neuen spezialisierten Datensatz angewendet, der aus grafischen Darstellungen von Straßennetzen in 22 Städten besteht. Um die Ergebnisse dieser Art von Analyse zu diskutieren, enthalten die Datensätze eine Reihe leicht identifizierbarer Straßennetzmodelle.

Die experimentellen Ergebnisse zeigen, dass der in dieser Studie gewählte Ansatz nicht nur eine praktische Methode zur Identifizierung von Ähnlichkeiten in verschiedenen grafisch dargestellten Straßennetzen basierend auf ihren topologischen Merkmalen darstellt, sondern auch Ähnlichkeiten und Korrelationen im Verhalten der zur Identifizierung von Ähnlichkeiten verwendeten Metriken aufzeigt. in Straßennetzen eingesetzt.

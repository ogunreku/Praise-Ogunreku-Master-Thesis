%%%%%%%%%%%%
% Abstract %
%%%%%%%%%%%%

Understanding the similarities of city road network structures is a difficult but necessary task for designing better urban environments. The discovery and comparison of structures such as graphs, modular communities, rich clubs, hubs, and trees provide information about the spatial and temporal properties of road networks.

With recent advancements in GIS and graph theory, and the rapid growth of available data and the flexibility of network modeling, the problem of developing effective quantitative and guaranteed a priori notions of metrics for comparing road networks has arisen. Numerous metrics exist to accomplish this task, the majority of which are applied to various fields, including bioinformatics, cyber security, and social network analysis. However, no comparative study has been conducted to determine the efficacy of these metrics in distinguishing common road network topologies at various structural scales.

The purpose of this thesis is to explore and compare the most frequently used metrics and distance measures, and assess their ability to distinguish between common topological features found in road networks. Finally, the metrics are applied to a novel specialized dataset consisting of graph representations of road networks in 22 cities. To discuss the results of this type of analysis, the datasets contain a collection of easily identifiable road network patterns.

Experimental results demonstrate that the approach adopted in this study not only provides a practical method of identifying similarities in different road networks represented as graphs based on their topological characteristics, but it also identifies similarities and correlations in the behavior of the metrics that are used to identify the similarities in the road networks.

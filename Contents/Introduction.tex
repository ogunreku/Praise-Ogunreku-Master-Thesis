%%%%%%%%%%%%%%%%
% Introduction %
%%%%%%%%%%%%%%%%

The emerging geographic information system (GIS) repositories, such as EarthExplorer or OpenStreetMap (OSM), provide publicly accessible high-quality road network data and cadastre data that can be used to feed example-based urban modeling techniques with almost infinite examples. However, recognizing the similarities of patterns between the designs of several urban layouts remains a difficult job, even with a large amount of high-quality example data at hand. The key explanations are;  there are no guaranteed a-priori notion of metrics that can recognize the similar patterns in road networks and quantify the input of a given consumer into a realistic city layout that includes road networks.  Current modeling tools-sets have a steep learning curve and will require special education skills to work with them productively. Current automated methods rely on a set of rules and grammar to generate urban structures; however, their expressiveness is constrained by the collection of rules. Expert skills are required to effectively define the typeset rules and, in many cases, new rules-sets need to be developed for each new road network style created. In order to make it possible for non-expert users to create urban structures for individual experiments, this thesis proposes a portfolio of novel example-based metrics for managed evaluation of similar road networks for urban environments. The notion example-based implies that new virtual urban environments are generated by computer programs that re-use existing digitized real-world data that serve as models. The data, i.e., road networks, required to realize the envisioned task, are publicly accessible via online services. In order to enable the reuse of existing urban datasets, it is important to establish quantitative and qualitative metrics that encapsulate expert knowledge and thus enable the controlled evaluation of the similarity between different road network patterns.  

The focus of this thesis is to compare the similarity of various road network patterns in 21 major cities around the world using OpenStreetMap data. It proposes a series of a priori notions of available metrics in the literature that encapsulate the qualitative and quantitative characteristics required for the evaluation of the similarities between road networks as one of the fundamental structures common to urban environments and the best measure of similarity (a real number between 0 and 1) that captures the similarity of the two road networks as graphs well. Similarity measures are used to investigate the similarities and differences between these study sites across multiple dimensions.

The rest of this thesis is organized as follows: Chapter 2 briefly will provide the background on urban forms, past and current techniques for the detection of road network patterns, the basics of graph structure for representing road networks, and existing metrics used in literature for identifying similarities in graph networks. Chapter 3 discusses the data used in this approach  and presents the methodology of the thesis. Chapter 4 describes the results of the employed methodology on the data. Finally, chapter 5 presents Conclusion and recommendation for future work. 



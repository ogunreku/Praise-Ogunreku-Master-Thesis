%%%%%%%%%%%%%%%
% Conclusions %
%%%%%%%%%%%%%%%

As long as there are functional/structural brain networks, there will be people looking for comparisons between them. Here, we have presented the main methods and algorithms that can be used to compare brain networks.

Which method then? The answer depends on the application itself. If the objective is to reveal statistical difference between two groups (healthy subjects vs. patients, for instance) then the methods based on the graph theoretical approach (node-wide or edge-wise) can be good candidates provided that physiological hypothesis are well set and statistical parameters are carefully chosen (correction for multiple comparisons, for instance).

However, if the objective is to produce a similarity score (usually normalized between 0 and 1), then the distance-based graph comparison methods are more appropriate. Validation of algorithms, like the comparative analysis of Mheich et al. (2018), can allow us to identify a set of methods that perform well on simulated networks. However, we do not know how well real networks are described by currently used simulations.
Therefore, there is no guarantee that methods performing well on benchmarks also give reliable results on real brain networks (advantages and limitations of some of these algorithms are presented in Table 4).

From an application viewpoint, the network similarity is crucial in the identification of what we called here “network of networks.” This can be used to build a “semantic map” where nodes can represent the estimated networks of visual/auditory objects and edges can denote the similarity between these networks (preliminary results are presented in this review). This will undoubtedly require a very large number of stimuli and also the repetition of each stimuli several times. When these conditions are respected, these semantic maps can give new insights into the object categorization process in the human brain, from a network-based perspective.
In clinical neuroscience, a potential application of network distance measures is the mapping of a “disease network” where the nodes may represent each brain disease and the edges can represent the similarity between the different networks associated to each disease (such as Parkinson’s, Alzheimer’s, epilepsy, and so on). This application could help to further understand the possible common altered network patterns in brain disease. A very recent review by van den Heuvel and Sporns (2019a) showed indeed the importance of investigating such cross-disorder connectivity patterns.
Another potential application of the network of networks approach is to construct a similarity network across species connectomes (van den Heuvel et al., 2016), in which nodes can denote species and edges the similarity between them. The major difficulty of this application is to have access to connectome data from a range of species (human, drozophila, C. elegans, cat, macaque, pigeon, mouse, rat, etc.). This may help to better understand cross-species communalities and differences in term of brain structure and function. Moreover, a combination of hypothesis-based selection of graph metrics (specific hubs, modules, etc.) with the similarity algorithms may also improve the categorization and the classifications of these subnetworks.


\section{Some Challenges in Road Network Similarity}
First, in the statistical comparison (graph theoretical–based approach), the major difficulty arises from the fact that graph measures depend on the number of nodes and edges. To compare two different brain networks, choosing equal size and density has become more popular so that differences in graph measures appear solely through structural changes (van Wijk, Stam, & Daffertshofer, 2010). However, this can be only achieved by taking a fixed number of nodes and imposing a desired average degree by adjusting the binary threshold (van Wijk et al., 2010).

Second, knowing the graph metrics that enable one to detect the difference between brain networks is not obvious. The choice of this graph metric is often empirical. For a more appropriate approach, these graph metrics should be driven by the physiopathology of the analyzed neuroscience question or by adopting methods based on Network Representation Learning (NRL). Indeed, these NRL approaches avoid the necessity for thorough feature engineering and have led to very important results in network-based tasks, such as node classification, node clustering, and prediction (D. Zhang, Yin, Zhu, & Zhang, 2018). We believe that NRL could be very useful to the network neuroscience community for the adaption of representation learning techniques to specific applications that are of interest in the field. In addition, a key challenge is to encapsulate several graph metrics in one similarity score that can describe the difference between two graphs without missing any characteristics.
Third, the distance-based graph algorithms developed in different fields are indeed very useful for detecting uncaptured characteristics by graph metrics. However, these algorithms are still limited to undirected graphs. More efforts are needed to expand these algorithms to deal with directed (causal) brain networks.


\section{Future Directions}
From a methodological viewpoint, more efforts are needed to develop and optimize similarity algorithms that combine several graph characteristics into one similarity score. These algorithms should be first analyzed/validated using simulated data (where ground truth can be, to some extent, obtained) before applying them to real brain networks. Another methodological approach is to use similarity methods in brain dynamic algorithms in order to decipher how brain networks change over time. In most dynamic analysis algorithms, a similarity/correlation step is always needed to compare adjacent networks. This is usually done by using the classical correlation coefficient (O’Neill et al., 2018). Adding a network-based similarity index into brain dynamic algorithms can dramatically improve their performance.
From an application viewpoint, an interesting future clinical application is the construction of a “network of brain diseases,” where nodes can represent brain diseases and edges represent the similarity score between them. This map may help to characterize and visualize the common landscapes between brain disorders, an issue recently reviewed by van den Heuvel and Sporns (2019b).
We hope that the survey will motivate more researchers to contribute with other ideas than those described above in the brain network similarity field, from a methodological and/or an applicative perspective.

\begin{example}
Here is a fancy example. Here is a fancy example. Here is a fancy example. Here is a fancy example. Here is a fancy example. Here is a fancy example. Here is a fancy example. Here is a fancy example. 
\end{example}                  